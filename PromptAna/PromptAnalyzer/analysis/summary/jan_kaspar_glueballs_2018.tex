\input slides.tex
%\input utf8-t1

%----------------------------------------------------------------------------------------------------

\def\author{J.~Kaspar}
\def\caption{Glueballs 2018}
\def\date{28 Jan 2019}



\newpage %-------------------------------------------------------------------------------------------
\title{General notes}

\> analysis code
\>> original code by Robert Ciesielski
\>> updated stored in https://github.com/jan-kaspar/analysis\_glueballs\_2018

\> data source
\>> PromptReco-v2, AOD
\>> runs included: 319176, 319256, 319262, 319263, 319265, 319268, 319300, 319311

\> triggers
\>> TOTEM2: diagonal configuration with elastic cut
\>> TOTEM4: anti-diagonal configurations (top-top and bottom-bottom)

\> upcoming slides: about $75\un{\%}$ of ntuples processed
\>> typical failure: input AOD file not available via XROOT/AAA


\newpage %-------------------------------------------------------------------------------------------
\title{Standard event selection}

\> general
\>> single vertex
\>> single (exclusive) RP configuration (left/right and top/bottom)
\>> anti-elastic cut by scattering angle comparison (see slide 3)

\> 4-track sample
\>> 4 tracks
\>> charge balance
\>> 4 tracks associated with the single vertex
\>> CMS-TOTEM RP matching by momentum comparison (see slides 4 and 5)


\newpage %-------------------------------------------------------------------------------------------
\title{Anti-elastic cut}

\> scattering angles reconstructed in elastic-like approximation

\> cuts applied: single vertex, single RP configuration

\centerline{\fig[14cm]{../plots/general/elastic_cut.pdf}}

\>> excluded region: between vertical green lines


\newpage %-------------------------------------------------------------------------------------------
\title{CMS-TOTEM RP matching}

\> comparison of transverse momenta from CMS and TOTEM RPs

\> cuts applied: single vertex, single RP configuration, anti-elastic cut, 4 tracks, 4 tracks associated with the vertex, charge balance

\centerline{\fig[12cm]{../plots/general/cms_totem_matching_2D.pdf}}


\newpage %-------------------------------------------------------------------------------------------
\title{CMS-TOTEM RP matching}

\> as on previous slide, but plotting CMS-TOTEM difference

\centerline{\fig[12cm]{../plots/general/cms_totem_matching.pdf}}

\>> selected region: between vertical green lines


\newpage %-------------------------------------------------------------------------------------------
\title{PID based on $\d E/\d x$}

\> ``standard cuts`` in 4-track sample (slide 2)

\centerline{\fig[,8.5cm]{../plots/general/dEdx.pdf}}

\>> kaon selection: between the black curves
\>> pion selection: below the dashed magenta curve


\newpage %-------------------------------------------------------------------------------------------
\title{Phi-Phi}

\> each track: kaon mass hypothesis

\> cuts
\>> ``standard event selection'' (see slide 2)
\>> mass of each $K^+ K^-$ pair compatible with $\ph$
\>>> both kaon combinations considered
\>>> central value $m(\ph) = 1.02\un{GeV}$, ``window'' width $\si = 0.02\un{GeV}$
\>>> different tolerances tried: ``$\si$'', ``$\si/2$''
\>> given number of tracks identified as kaons with $\d E/\d x$ (see slides 9 and 11)


\newpage %-------------------------------------------------------------------------------------------
\ctitle{Phi-Phi}{Kaon-pair mass distributions, $\si$ cut}

\> plots of invariant masses of kaon pairs
\>> kaons combined such that total charge is zero
\>> $m_{ij}$: combination of $i$-th positive and $j$-th negative kaon

\centerline{\fig[15cm]{../plots/phi-phi/mass_pair_dist.pdf}}

\>> selection regions: black square of green axes



\newpage %-------------------------------------------------------------------------------------------
\ctitle{Phi-Phi}{4-kaon mass distributions, $\si$ cut}

\centerline{\fig[15cm]{../plots/phi-phi/mass_dist.pdf}}



\newpage %-------------------------------------------------------------------------------------------
\ctitle{Phi-Phi}{Kaon-pair mass distributions, $\si/2$ cut}

\> plots of invariant masses of kaon pairs
\>> kaons combined such that total charge is zero
\>> $m_{ij}$: combination of $i$-th positive and $j$-th negative kaon

\centerline{\fig[15cm]{../plots/phi-phi/mass_pair_dist_PhiCutStrict.pdf}}

\>> selection regions: black square of green axes



\newpage %-------------------------------------------------------------------------------------------
\ctitle{Phi-Phi}{4-kaon mass distributions, $\si/2$ cut}

\centerline{\fig[15cm]{../plots/phi-phi/mass_dist_PhiCutStrict.pdf}}



\newpage %-------------------------------------------------------------------------------------------
\title{Rho-Rho}

\> each track: pion mass hypothesis

\> cuts
\>> ``standard event selection'' (see slide 2)
\>> mass of each $\pi^+ \pi^-$ pair compatible with $\rh$
\>>> both pion combinations considered
\>>> central value $m(\rh) = 0.77\un{GeV}$, ``window'' width $\si = 0.124\un{GeV}$
\>>> different tolerances tried: ``$\si$'', ``$\si/2$'' and ``$\si/4$''
\>> given number of tracks identified as pions with $\d E/\d x$ (see slides 14 and 15)


\newpage %-------------------------------------------------------------------------------------------
\ctitle{Rho-Rho}{Pion-pair mass distributions, $\si$ cut}

\> plots of invariant masses of pion pairs
\>> pions combined such that total charge is zero
\>> $m_{ij}$: combination of $i$-th positive and $j$-th negative pion

\centerline{\fig[15cm]{../plots/rho-rho/mass_pair_dist_cut1.pdf}}

\>> selection regions: black square of green axes


\newpage %-------------------------------------------------------------------------------------------
\ctitle{Rho-Rho}{4-pion mass distributions, $\si/2$ cut}

\centerline{\fig[16cm]{../plots/rho-rho/mass_dist_cut2.pdf}}


\newpage %-------------------------------------------------------------------------------------------
\ctitle{Rho-Rho}{4-pion mass distributions, $\si/4$ cut}

\centerline{\fig[16cm]{../plots/rho-rho/mass_dist_cut4.pdf}}

\vfil
\eject
\bye
